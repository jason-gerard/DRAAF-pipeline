%---------------------------------------------------------
\chapter{Introduction}
%---------------------------------------------------------

In recent years, the deployment of 5G/Beyond 5G and C-V2X (Cellular Vehicle-to-Everything) networks has paved the way for deploying real-time vehicular safety applications \cite{ul2019green}. Thanks to the reduction of delays and the appearance of different architectures, information can be easily and safely exchanged between different actors on the traffic routes \cite{tudzarov2011functional}. C-V2X technology allows different modes of communication to connect cars and Vulnerable Road Users (VRU). 

%In fact, over C-V2X network, it is possible to dispense with equipment dedicated to establishing vehicle networks (as was formerly used in the IEEE 802.11p protocol), allowing smartphone’s use to constitute the network structure. 
In fact, over the C-V2X network, it is possible to establish communication between devices such as smartphones enabled to work with this network. This way, vulnerable users such as pedestrians, cyclists, and others can participate in the vehicular network. The inclusion of VRU was not straightforward over the IEEE 802.11p standard since network participants needed dedicated equipment for exchanging messages over the wireless network.


The usage of C-V2X through smartphones or other VRU devices (e.g., 5G-enabled smart helmets) also poses a series of challenges to consider. One of the main challenges is managing congestion over the communications network, as the number of devices connected to the network may increase significantly. The density of devices in crowded vehicular scenarios represents a challenge to the operation of the vehicular network. Recent proposals of architectures and device groupings tackle the device crowding problems, including over-signaling and networks collapsing \cite{hakeem20205g}.

This is a report of the literature review conducted on the topic of Collective Perception Service (CPS) and its relationship with VRU detection.%  a topic directly related to the doctoral thesis proposal.  
For this purpose, we reviewed recent articles published in journals and conferences of significant impact and recognized by the international scientific community. The remainder of this document is organized as follows: Chapter 2 presents the study of the literature agreed upon with the Advisor. Chapter 3 discusses and critically analyzes the possible research gaps that have not been covered or are posed as future challenges. Finally, Chapter 4 provides the work plan for the development of the doctoral thesis. 