%\chapter*{References}\label{ch:appendix_a}
%\addcontentsline{toc}{chapter*}{References}


% \begin{spacing}{1}
% \begin{thebibliography}{100}
% \addcontentsline{toc}{chapter}{References}

% \bibitem{story1}
% David R. Woolley (12 February 2013). 
% \emph{PLATO: The Emergence of Online Community},
  
% \bibitem{pippo}
% Crow, W. B. \& Din, H. (2009)
% \emph{ Unbound By Place or Time: Museums and Online Learning},
% Washington, DC: American Association of Museums, pp. 9-10

% \bibitem{story3}
% Graziadei, W. D., et al., 1997
% \emph{ Building Asynchronous and Synchronous Teaching-Learning Environments: Exploring a Course/Classroom Management System Solution},
% \url{http://horizon.unc.edu/projects/monograph/CD/Technological_Tools/Graziadei.html}


% \bibitem{texbook}
% Donald E. Knuth (1986) \emph{The \TeX{} Book}, Addison-Wesley Professional.

% \bibitem{lamport94}
% Leslie Lamport (1994) \emph{\LaTeX: a document preparation system}, Addison
% Wesley, Massachusetts, 2nd ed.


% \bibitem{Azuela} Mariano Azuela, \textit{The Underdogs: A Novel of the Mexican Revolution}, trans. Beth Jorgensen (New York: The Modern Library, 2002). 

% \bibitem{notes} John W. Dower {\em Readings compiled for History
%   21.479.}  1991.

%   \bibitem{impj}  The Japan Reader {\em Imperial Japan 1800-1945} 1973:
%   Random House, N.Y.

%   \bibitem{norman} E. H. Norman {\em Japan's emergence as a modern
%   state} 1940: International Secretariat, Institute of Pacific
%   Relations.

%   \bibitem{fo} Bob Tadashi Wakabayashi {\em Anti-Foreignism and Western
%   Learning in Early-Modern Japan} 1986: Harvard University Press.



% \end{thebibliography}
% \end{spacing}

%\chapter*{Bibliography}

